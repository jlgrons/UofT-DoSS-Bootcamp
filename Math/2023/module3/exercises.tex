\documentclass{article}
\usepackage[utf8]{inputenc}
\usepackage[margin=1in]{geometry}
\usepackage{hyperref}
\usepackage{setspace}
\pagenumbering{arabic}
\usepackage{graphicx}
\usepackage[dvipsnames]{xcolor}
\usepackage{fancyhdr} 
\usepackage{amsmath, amsfonts, amsthm, amssymb}
\usepackage{bbm}
\usepackage{nth}
\usepackage{dsfont}

\hypersetup{
  colorlinks   = true, %Colours links instead of boxes
  urlcolor     = black, %Colour for external hyperlinks
  linkcolor    = black, %Colour of internal links
  citecolor   = black %Colour of citations
}

\allowdisplaybreaks % fixes align environment weird spacing on page
\setlength{\parindent}{0cm}


%For references using Lemma, Theorem, etc, use \cref
\usepackage[nameinlink,capitalize,sort]{cleveref}

% theorems
\newtheorem{theorem}{Theorem}[section]
\newtheorem{lemma}[theorem]{Lemma}
\newtheorem{definition}[theorem]{Definition}
\newtheorem{proposition}[theorem]{Proposition}
\newtheorem{example}[theorem]{Example}
\newtheorem{corollary}[theorem]{Corollary}
%theoremstyle{plain} %boldface title, italicized body. Commonly used in theorems, lemmas, corollaries, propositions and conjectures.
%\theoremstyle{definition} %boldface title, Roman body. Commonly used in definitions, conditions, problems and examples
\theoremstyle{remark} %italicized title, Roman body. Commonly used in remarks, notes, annotations, claims, cases, acknowledgments and conclusions.
\newtheorem{exercise}[theorem]{Exercise}

\newenvironment{solution}
  {\renewcommand\qedsymbol{$\blacksquare$}\begin{proof}[Solution]}
  {\end{proof}}

% weird hack to get rid of dot:
\usepackage{xpatch}
\makeatletter
\AtBeginDocument{\xpatchcmd{\@thm}{\thm@headpunct{.}}{\thm@headpunct{}}{}{}}



% lin alg
\newcommand{\bu}{{\mathbf{u}}}
\newcommand{\bv}{{\mathbf{v}}}
\newcommand{\bw}{{\mathbf{w}}}
\newcommand{\bx}{\mathbf{x}}
\newcommand{\zerovec}{{\mathbf{0}}}

% other useful stuff
\newcommand{\Id}{{\mathds{1}}}
\newcommand{\R}{{\mathbb{R}}}
\newcommand{\C}{{\mathbb{C}}}
\newcommand{\Z}{{\mathbb{Z}}}
\newcommand{\N}{{\mathbb{N}}}
\newcommand{\Q}{{\mathbb{Q}}}
\newcommand{\F}{{\mathbb{F}}}
\newcommand{\cL}{{\mathcal{L}}}
\newcommand{\cP}{\mathcal{P}}
\newcommand{\cT}{\mathcal{T}}
\newcommand{\inv}{{-1}}

%\DeclareMathOperator{\dim}{dim}
\DeclareMathOperator{\range}{range}
\DeclareMathOperator{\rank}{rank}
\DeclareMathOperator{\nullity}{nullity}


\begin{document}
\begin{center}
\Large{Exercises for Module 3: Set Theory II and Metric Spaces I}
\end{center}

1. Show that for sets $A,B \subseteq X$ and $f: X \to Y$, $f(A \cap B) \subseteq f(A) \cap f(B)$.

\vspace{11cm} % delete this

% \begin{proof}  
% ADD your content 
% \end{proof}  


2. Let $f: X \to Y$ and $B \subseteq Y$. Prove that $f(f^{-1}(B)) \subseteq B$, with equality iff $f$ is surjective.

\vspace{13cm} % delete this

% \begin{proof}  
% ADD your content 
% \end{proof}  



3. Prove that $f(\cup_{i \in I}A_i) = \cup_{i \in I}f(A_i)$, where $f:X \to Y$, $A_i \subseteq X \, \forall i \in I$. 

\vspace{11cm} % delete this

% \begin{proof}  
% ADD your content 
% \end{proof}  



4. Show that $\N$ and $\Z$ have the same cardinality. 

\vspace{11cm} % delete this

% \begin{proof}  
% ADD your content 
% \end{proof}  



5. Show that $|(0,1)| =|(1,\infty)|$.

\vspace{11cm} % delete this

% \begin{proof}  
% ADD your content 
% \end{proof}  




6. Show that the infinity norm $||x||_\infty$, $x \in \R^n$, is a norm.

    
\vspace{11cm} % delete this

% \begin{proof}  
% ADD your content 
% \end{proof}  



7. Let $(X,d)$ be any metric space, and define $\tilde d: X \times X \to \R$ by 
    \begin{equation*}
        \tilde d(x,y) = \frac{d(x,y)}{1+d(x,y)}, \quad x,y \in X .
    \end{equation*}
    Show that $\tilde d$ is a metric on $X$.

\vspace{11cm} % delete this

% \begin{proof}  
% ADD your content 
% \end{proof}  


8. Let $X$ be a set and define $d\colon X \times X \to \R$ by $d(x,x) = 0$ and $d(x,y)=1$ for $x\neq y \in X$. Prove that $d$ is a metric on $X$. What do open balls look like for different radii $r>0$? %What does an arbitrary open set look like?

\vspace{13cm} % delete this

% \begin{proof}  
% ADD your content 
% \end{proof}  





\end{document}


