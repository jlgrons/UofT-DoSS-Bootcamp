\documentclass{article}
\usepackage[utf8]{inputenc}
\usepackage[margin=1in]{geometry}
\usepackage{hyperref}
\usepackage{setspace}
\pagenumbering{arabic}
\usepackage{graphicx}
\usepackage[dvipsnames]{xcolor}
\usepackage{fancyhdr} 
\usepackage{amsmath, amsfonts, amsthm, amssymb}
\usepackage{bbm}
\usepackage{nth}
\usepackage{dsfont}
\usepackage{subfig}
\usepackage{tikz}
\usepackage{accents}


% commenting
\newcommand{\comment}[3]{\textcolor{#1}{\textbf{[#2: }\textit{#3}\textbf{]}}}
\newcommand{\emma}[1]{\comment{purple}{EK}{#1}}
\newcommand{\jesse}[1]{\comment{BurntOrange}{JG}{#1}}
\newcommand{\miaoshiqi}[1]{\comment{ForestGreen}{ML}{#1}}
\newcommand{\siyue}[1]{\comment{blue}{SY}{#1}}

%For references using Lemma, Theorem, etc, use \cref
\usepackage[nameinlink,capitalize,sort]{cleveref}

% theorems
\newtheorem{theorem}{Theorem}[section]
\newtheorem{lemma}[theorem]{Lemma}
\newtheorem{definition}[theorem]{Definition}
\newtheorem{proposition}[theorem]{Proposition}
\newtheorem{example}[theorem]{Example}
\newtheorem{corollary}[theorem]{Corollary}
%theoremstyle{plain} %boldface title, italicized body. Commonly used in theorems, lemmas, corollaries, propositions and conjectures.
%\theoremstyle{definition} %boldface title, Roman body. Commonly used in definitions, conditions, problems and examples
\theoremstyle{remark} %italicized title, Roman body. Commonly used in remarks, notes, annotations, claims, cases, acknowledgments and conclusions.
\newtheorem{exercise}[theorem]{Exercise}

\newenvironment{solution}
  {\renewcommand\qedsymbol{$\blacksquare$}\begin{proof}[Solution]}
  {\end{proof}}

% weird hack to get rid of dot:
\usepackage{xpatch}
\makeatletter
\AtBeginDocument{\xpatchcmd{\@thm}{\thm@headpunct{.}}{\thm@headpunct{}}{}{}}



% lin alg
\newcommand{\bu}{{\mathbf{u}}}
\newcommand{\bv}{{\mathbf{v}}}
\newcommand{\bw}{{\mathbf{w}}}
\newcommand{\bx}{\mathbf{x}}
\newcommand{\zerovec}{{\mathbf{0}}}

% other useful stuff
\newcommand{\Id}{{\mathds{1}}}
\newcommand{\N}{{\mathbb{N}}}
\newcommand{\R}{{\mathbb{R}}}
\newcommand{\C}{{\mathbb{C}}}
\newcommand{\Z}{{\mathbb{Z}}}
\newcommand{\Q}{{\mathbb{Q}}}
\newcommand{\F}{{\mathbb{F}}}
\newcommand{\cL}{{\mathcal{L}}}
\newcommand{\cP}{\mathcal{P}}
\newcommand{\inv}{{-1}}

%\DeclareMathOperator{\dim}{dim}
\DeclareMathOperator{\range}{range}
\DeclareMathOperator{\rank}{rank}
\DeclareMathOperator{\nullity}{nullity}

\hypersetup{
  colorlinks   = true, %Colours links instead of boxes
  urlcolor     = blue, %Colour for external hyperlinks
  linkcolor    = black, %Colour of internal links
  citecolor   = black %Colour of citations
}

\allowdisplaybreaks % fixes align environment weird spacing on page
\setlength{\parindent}{0cm}
\usepackage[parfill]{parskip}

% get rid of dot after theorem 
\usepackage{xpatch}
\makeatletter
% Patch to accommodate for \begin{theorem}[...]
\AtBeginDocument{\xpatchcmd{\cref@thmoptarg}{\thm@headpunct{.}}{\thm@headpunct{}}{}{}}
% Patch to accommodate \begin{theorem} (without an optional argument)
\AtBeginDocument{\xpatchcmd{\cref@thmnoarg}{\thm@headpunct{.}}{\thm@headpunct{}}{}{}}

% \usepackage[natbib=true, style=vancouver]{biblatex}
\usepackage[backend= biber, style=alphabetic]{biblatex}
 %\usepackage{natbib}
 %\setcitestyle{square}
\bibliography{references.bib}

\title{Mathematics Bootcamp Lecture Notes \\
\vspace{0.5em}
\large Department of Statistical Sciences, University of Toronto}
\author{Emma Kroell}
\date{Last updated: July 9, 2023}

\begin{document}

\maketitle


\newpage
\section*{Preface}

These lecture notes were prepared for the Operational Mathematics course at the Department of Statistical Sciences Graduate Student Bootcamp at the University of Toronto, held in July 2022 and July 2023. The course teaches an overview of necessary mathematics prerequisites to incoming statistics graduate students, with an emphasis on proofs.


These lectures are based on the following books:

1. \href{https://link-springer-com.myaccess.library.utoronto.ca/book/10.1007/978-1-4614-4265-3}{{\emph{An Introduction to Mathematical Structures and Proofs}}} by Larry J. Gerstein \\
2. \href{https://bookstore.ams.org/amstext-26/}{{\emph{The Tools of Mathematical Reasoning}}} by Tamara J. Lakins \\
3. \href{https://link-springer-com.myaccess.library.utoronto.ca/book/10.1007/0-387-28387-0}{\emph{A Taste of Topology}} by Volker Runde \\
4. \href{https://link-springer-com.myaccess.library.utoronto.ca/book/10.1007/978-3-319-17771-7}{\emph{Real Mathematical Analysis}} by Charles C. Pugh \\
5. \href{https://link-springer-com.myaccess.library.utoronto.ca/book/10.1007/978-3-319-11080-6}{{\emph{Linear Algebra Done Right}}} by Sheldon Axler \\
6. \href{https://www.math.brown.edu/streil/papers/LADW/LADW.html}{{\emph{Linear Algebra Done Wrong}}} by Sergei Treil 

Chapter 1 of Gerstein (2012) and the first three chapters of Lakins (2016) are used as references for the proof technique section. Runde (2005) is the main text for the sections on set theory, metric spaces, and topology, which follow chapters 1, 2, and 3 of his book, respectively. Some additional topics come from Pugh (2015). The linear algebra content comes mostly from Axler (2015), with Treil (2017) used in some sections for an alternate perspective, particularly for the content on determinants and spectral theory. The content on differentiation and integration is based on Chapter 3 of Pugh (2015). A few other texts and lecture notes are used at certain points, and are cited in the references section corresponding to that chapter.

Most of the material in these notes belongs to these texts. Many of these texts are available online to University of Toronto users (some to everyone).

I would like to acknowledge the assistance of Jesse Gronsbell, Stanislav Volgushev, Piotr Zwiernik, and Robert Zimmerman in developing the list of topics for the course. 

Please notify me of any typos or corrections at emma.kroell@mail.utoronto.ca.


\newpage
\tableofcontents


\newpage

{\bf A short note on notation: }

\vspace{0.5em}

$\N$ denotes the whole numbers, i.e. $\N = \{1,2,\ldots \}$

$\N_0$ denotes the non-negative integers, i.e. $\N_0 = \{0, 1,2,\ldots \}$

$\Z$ denotes the integers, i.e. $\Z = \{ \ldots, -2,-1,0,1,2,\ldots \}$

$\Q$ denotes the rational numbers, i.e. $\Q = \{ \frac{p}{q} \, | \, p, q \in \Z \text{ and } q \neq 0 \}$

$\R$ denotes the real numbers

$\C$ denotes the complex numbers



\section{Review of proof techniques}
\subsection{Propositional logic}

\emph{Propositions} are statements that could be true or false. They have a corresponding \emph{truth value}. We will use capital letters to denote propositions. For example, ``$n$ is odd'' and ``$n$ is divisible by 2'' are propositions. Let's call them $P$ and $Q$. Whether they are true or not (i.e. their truth value) depends on what $n$ is. 

We can  negate statements: $\neg P$ is the statement ``$n$ is not odd''.

 We can combine statements: 
 \begin{itemize}
 \item $P \wedge Q$ is the statement ``$n$ is odd and $n$ is divisible by 2''.
 \item $P \vee Q$ is the statement ``$n$ is odd or $n$ is divisible by 2''.
\end{itemize}
We always assume the inclusive or unless specifically stated otherwise.


\begin{example}
Here are some statements, which we want to write in propositional logic.
\begin{itemize}
              \item If it's not raining, I won't bring my umbrella.
              \item I'm a banana or Toronto is in Canada.
              \item If I pass this exam, I'll be both happy and surprised.
\end{itemize}

For the first one, let $A$ be the statement ``it's raining'' and $B$ be the statement ``I will bring my umbrella''. In logic, the statement is $\neg A \implies \neg B$.

For the second, let $C$ be the statement ``I'm a banana'' and let $D$ be the statement ``Toronto is in Canada''. We write this as $C \vee D$.

For the third, let $P$ be the statement ``I pass this exam'', let $Q$ be the statement ``I am happy'', and let $R$ be the statement ``I am surprised''. This one is written $P \implies (Q \wedge R)$.
\end{example}

\subsubsection{Truth values}

\begin{example} Write the following using propositional logic: \\
If it is snowing, then it is cold out. \\
It is snowing. \\
Therefore, it is cold out.

Solution. \\
$P \implies Q$ \\
$P$ \\
Conclusion: $Q$ \\
\end{example}

To examine if a statement is true or not, we use a truth table, where we write out all the possibilities. 

\begin{example}
The truth table for $P \implies Q$ where $P$, $Q$ are propositions is:

\begin{tabular}{|c|c| c|}
\hline
     $P$& $Q$ &  $P \implies Q$ \\ \hline
     T& T & T \\ \hline
     T & F & F \\ \hline
     F & T & T \\ \hline
     F & F & T \\ \hline
\end{tabular}
\end{example}

\subsubsection{Logical equivalence}
We say that two statements are \emph{logically equivalent} if they have the same truth tables.

\begin{example}
\label{ex:implication_table}
Let $P$, $Q$ be propositions. $P \implies Q$ is logically equivalent to $\neg P \vee Q$.
 
 
\begin{tabular}{|c|c| c|}
\hline
     $P$& $Q$ &  $P \implies Q$ \\ \hline
     T& T & T \\ \hline
     T & F & F \\ \hline
     F & T & T \\ \hline
     F & F & T \\ \hline
\end{tabular} \hspace{2cm} \begin{tabular}{|c | c | c | c|}
\hline
     $P$& $Q$ & $\neg P$ & $\neg P \vee Q$  \\ \hline
     T& T & F & T \\ \hline
     T & F & F & F \\ \hline
     F & T &  T &T \\ \hline
     F & F & T & T \\ \hline
\end{tabular}
\end{example}

\vspace{1em}

\begin{theorem}[De Morgan's Laws]
\label{thm:demorgan}
Let $P$, $Q$ be propositions.
\begin{enumerate}
    \item[(i)] $\neg (P \wedge Q)$ is logically equivalent to $\neg P \vee \neg Q$.
    \item[(ii)] $\neg (P \vee Q)$ is logically equivalent to $\neg P \wedge \neg Q$.
\end{enumerate}
\end{theorem}

Proving this is your first exercise.

The following fact is often useful.
\begin{example}
\label{ex:ifthen}
$\neg (P \implies Q)$ is logically equivalent to $P \wedge \neg Q$. This follows from \cref{ex:implication_table} and \cref{thm:demorgan}.
\end{example}

\subsubsection{Quantifiers}
There are two important logical operators that we have not yet discussed. They are denoted using the following symbols: $\forall$, read as ``for all'' or ``for each'', and $\exists$, read as ``there exists''. We will explore their meanings, how they can help us simplify statements we need to prove, and how we prove such statements.

\textbf{For all}

``for all'', $\forall$, is also called the universal quantifier. If $P(x)$ is some property that applies to $x$ from some domain, then $\forall x P(x)$ means that the property $P$ holds for every $x$ in the domain. An example is the statement ``Every real number has a non-negative square.'' We write this as $\forall x \in \R, \, x^2 \geq 0$. In logic, people often use brackets to separate parts of the logical expression, ex. $(\forall x \in \R) (x^2 \geq 0)$.

How do we prove a for all statement? We need to take an arbitrary element of the domain, and show the property holds for that element.

\textbf{There exists}

``there exists'', $\exists$, is also called the existential quantifier. If $P(x)$ is some property that applies to $x$ from some domain, then $\exists x P(x)$ means that the property $P$ holds for some $x$ in the domain. An example is the statement that 4 has a square root in the reals. We write this as $\exists x \in \R$ such that $x^2 = 4$ or in proper logic notation, $(\exists x \in \R$) ($x^2 = 4)$.

How do we prove a there exists statement? We need to find an element in the domain for which the property holds (find an example).

There is also a special way of writing when there exists a unique element. We use $\exists!$ for this case. For example, the statement ``there exists a unique positive integer such that the integer squared is 64'' is written $\exists!  z \in \N$ such that $z^2 = 64$.

\textbf{Combining quantifiers}

Often we will need to prove statements where we combine quantifiers. Here are some examples:

\begin{tabular}{p{0.45\textwidth} p{0.45\textwidth}}
     Statement & Logical expression \\
     \hline
     Every non-zero rational number has a multiplicative inverse & $\forall q \in \Q \setminus \{0\}, \, \exists s \in \Q$ such that $qs=1$ \\
     Each integer has a unique additive inverse & $\forall x \in \Z \, , \exists ! y \in \Z$ such that $x+y = 0$ \\
     $f:\R \to \R$ is continuous at $x_0\in\R$ \ &  $\forall \epsilon >0 \; \exists \delta > 0$ such that whenever $|x - x_0| < \delta$, $|f(x)-f(x_0)| < \epsilon$
\end{tabular}

\vspace{1em}

The order of quantifiers is important! Changing the order changes the meaning. Consider the following example. Which are true? Which are false?

\begin{center}

$\forall x \in \R \, \forall y \in \R$  $x + y = 2$ \\
$\forall x \in \R \, \exists y \in \R$  $x + y = 2$ \\
$\exists x \in \R \, \forall y \in \R$  $x + y = 2$ \\
$\exists x \in \R \, \exists y \in \R$ $x + y = 2$ 
\end{center}
\vspace{1em}

It's also important to know how to negate logical statements that include quantifiers, as it will often help us prove or disprove the statements. The results are intuitive, but things can get complicated when we have more complex statements. The negation of a statement being true for all $x$ is that is isn't true for at least one $x$. The negation of a statement being true for at least one $x$ is that is isn't true for any $x$. 

In summary,

\begin{center}
    $\neg \forall x P(x)$ = $\exists  x (\neg P(x))$ \\
$\neg \exists x P(x)$ = $\forall x (\neg P(x))$
\end{center}

\vspace{1em}

The negations of the statements above are:
\vspace{1em}

\begin{tabular}{p{0.45\textwidth} p{0.45\textwidth}}
     Logical expression & Negation \\
     \hline
     $\forall q \in \Q \setminus \{0\}, \, \exists s \in \Q$ such that $qs=1$ & $\exists q \in \Q \setminus \{0\}$ such that  $\forall s \in \Q, \, qs \neq 1$\\
     $\forall x \in \Z \, , \exists ! y \in \Z$ such that $x+y = 0$ & $\exists x \in \Z$ such  that $(\forall y \in \Z, x+y \neq 0)$ $\vee$ $(\exists y_1, y_2 \in \Z$ such that $y_1 \neq y_2 \wedge x+y_1 = 0 \wedge x+y_2 = 0$  ) \\
     $\forall \epsilon >0 \; \exists \delta > 0$ such that whenever $|x - x_0| < \delta$, $|f(x)-f(x_0)| < \epsilon$ & $\exists \epsilon >0$ such that $\forall \delta > 0$,  $|x - x_0| < \delta$ and  $|f(x)-f(x_0)| \geq \epsilon$
\end{tabular}

\vspace{1em}

Note that we use De Morgan's laws (\cref{thm:demorgan}), as well as the negation of an implication (\cref{ex:ifthen}). What do these negations mean in English?


\subsection{Types of proof}

\subsubsection{Direct proof}

In a direct proof, our approach is to use the definition and known results.

\begin{example}
The product of an even number with another integer is even.
\end{example}

To prove this statement, we will use the definition of even. First we state that definition.

\begin{definition}
We say that an integer $n$ is \emph{even} if there exists another integer $j$ such that $n=2j$. \\
We say that an integer $n$ is \emph{odd} if there exists another integer $j$ such that $n=2j+1$.
\end{definition}

Now we prove the example directly.

\begin{proof}
Let $n, m \in \Z$, with $n$ even. By definition, there $\exists$ $j \in \Z$ such that $n = 2j$. Then 
$$ n m  =  (2 j) m = 2 (j m)$$
Therefore $n m$ is even by definition. 
\end{proof}

Here is another example, which uses the concept of divisibility.

\begin{definition}
Let $a,b \in \Z$. We say that ``a divides b'', written $a | b$, if the remainder is zero when $b$ is divided by $a$, i.e. $\exists j \in \Z$ such that $b = a j$.
\end{definition}

% https://hsm.stackexchange.com/questions/5656/who-invented-the-divisibility-symbol-and-why-is-it-backwards

\begin{example}
Let $a,b,c \in \Z$ with $a \neq 0$. Prove that if $a | b$ and $b | c$, then $a | c$.
\end{example}
\begin{proof}
Let $a,b,c \in \Z$. Suppose $a | b$ and $b | c$. Then by definition, there exists $j,k \in \Z$ such that $b = aj$ and $c = kb$. Combining these two equations gives $c = k (aj) = a (kj)$. Thus $a | c$ by definition.
\end{proof}


\subsubsection{Proof by contrapositive}
Sometimes instead of proving an implication $P \implies Q$ directly, it is easier to prove $\neg Q \implies \neg P$. This is called the contrapositive. First, we show that these two statements are logically equivalent using truth tables. 

$P \implies Q$  \hspace{5cm}  $\neg Q \implies \neg P$

        \vspace{1em}
\begin{tabular}{|c|c| c|}
\hline
     $P$& $Q$ &  $P \implies Q$ \\ \hline
     T& T & T \\ \hline
     T & F & F \\ \hline
     F & T & T \\ \hline
     F & F & T \\ \hline
\end{tabular}   \hspace{2cm}  \begin{tabular}{|c | c | c |  c | c |}
\hline
     $P$& $Q$ & $\neg P$ &  $\neg Q$ & $\neg Q \implies \neg P$ \\ \hline
     T& T & F & F & T \\ \hline
     T & F & F &  T & F \\ \hline
     F & T &  T  & F & T \\ \hline
     F & F & T & T & T \\ \hline
\end{tabular}
\vspace{1.5em}

Note that $\neg P \implies \neg Q$ is \emph{not} logically equivalent to  $P \implies Q$ (can you think of an example?). This is a common mistake.

Here is an example of a statement that is easier to prove using the contrapositive as opposed to directly.

\begin{example}
If an integer squared is even, then the integer is itself even.
\end{example}

\begin{proof}
We prove the contrapositive. Let $n$ be odd. Then there exists $k \in \Z$ such that $n = 2k + 1$. We compute
$$n^2 = (2k + 1)^2 = 4k^2 + 4k + 1 = 2(2k^2+2k) + 1.$$
Thus $n^2$ is odd.
\end{proof}


\subsubsection{Proof by contradiction}

Another proof technique is to assume something we know to be (or think to be) false, and then try to derive a contradiction. A contradiction is something that is impossible, like 0=1 or showing that a number is both odd and even. 

In sum, to prove that a statement $P$ is true by contradiction, we assume $\neg P$ is true, derive a contradiction, and conclude that $P$ is true. Here is an example.

\begin{example}
The sum of a rational number and an irrational number is irrational.
\end{example}

\begin{proof}
Let $q \in \mathbb{Q}$ and $r \in \mathbb{R} \setminus \mathbb{Q}$.
Suppose in order to derive a contradiction that their sum is rational, i.e. $ r + q = s$ where $s \in \mathbb{Q}$.
But then $r = s - q \in \mathbb{Q}$. Contradiction. Therefore the sum of a rational number and an irrational number is irrational.
\end{proof}


\subsubsection{Summary}

{\bf In sum, to prove $P \implies Q$:} 

\begin{tabular}{r l}
     Direct proof:  & assume $P$, prove $Q$ \\
     Proof by contrapositive:  & assume $\neg Q$, prove $\neg P$ \\ 
     Proof by contradiction: & assume $P \wedge \neg Q$ and derive something that is impossible \\ 
\end{tabular}

\vspace{0.5em}

\subsubsection{Induction}

Finally, we consider a special proof technique for proving statements about the natural numbers (or subsets of them of certain forms). It is based on the following theorem, which we state without proof. 

\begin{theorem}[Well-ordering principle for $\mathbb{N}$]
Every nonempty set of natural numbers has a least element.
\end{theorem}

Because of the well-ordering principle, we can prove something holds for the natural numbers by proving it holds for the smallest one, and then creating a logical ladder linking them together as follows:

\begin{theorem}[Principle of mathematical induction]
Let $n_0$ be a non-negative integer. Suppose $P$ is a statement about positive integers $n$ such that 
\begin{enumerate}
\item(base case) $P(n_0)$ is true 
\item (induction step) For every integer $k \geq n_0$, if $P(k)$ is true, then $P(k+1)$ is true.
\end{enumerate}
Then $P(n)$ is true for every integer $n \geq n_0$
\end{theorem}

Here is an example of a proof by induction.

\begin{example}
$n! > 2^n$ if $n \geq 4$.
\end{example}

\begin{proof}
We prove this by induction on $n$. \\
{\it Base case:} Let $n = 4$. Then $n! = 4! = 24 > 16 = 2^4$. \\
{\it Inductive hypothesis:} Suppose for some $k \geq 4$, $k! > 2^k$. \\
Then
$$(k+1)! = (k+1) k! > (k+1) 2^k > 2 (2^k) = 2^{k+1}.$$
Thus the statement holds by induction on $n$.
\end{proof}

Sometimes we use a different version of induction, called strong induction.  

\begin{theorem}[Principle of strong mathematical induction]
Let $n_0$ be a non-negative integer.  Suppose $P$ is a statement about positive integers $n$ such that
\begin{enumerate}
\item(base case) $P(n_0)$ is true 
\item (induction step) For every integer $k \geq n_0$, if $P(m)$ is true for every integer $m$ with $n_0 \leq m \leq k$, then $P(m+1)$ is true.
\end{enumerate}
Then $P(n)$ is true for every integer $n \geq n_0$.
\end{theorem}

Next, we will consider an example where it is much simpler to use the strong version of induction than the regular one. First, we recall the definition of a prime number.

\begin{definition}
\label{def:prime}
A positive integer $p$ is prime if $p$ has exactly two positive integer factors: 1 and $p$. Note that $1$ is not prime. We can write this as 
$$ p > 1 \text{ is prime if } \forall \, a, b \in \N, \, p=ab \implies (a = 1 \text{ or } b =1).$$
\end{definition}

We want to prove the existence part of the Fundamental Theorem of Arithmetic, that every integer greater than or equal to 2 has a prime factorization. The fact that such a factorization is unique is left as an exercise.

\begin{example}
Every integer $n \geq 2$ can be written as the product of primes.
\end{example}

\begin{proof}
We prove this using the Principle of Strong Mathematical Induction on $n$.

{\it Base case:} $n = 2$ is prime. 

{\it Inductive hypothesis:} Suppose for some $k \geq 2$ that one can write every integer $n$ such that $2 \leq n \leq k$ as a product of primes.

We must show that we can write $k+1$ as a product of primes.

{\it Case 1:} if $k+1$ is prime, then we are done. 

{\it Case 2:} if $k+1$ is not prime, then by \cref{def:prime}, there exists $a,b \in \N$ such that $k+1 = ab$ where $a,b \neq 1$. Then it must also be the case that $a,b \leq k$.

By the inductive hypothesis, we can write $a$ and $b$ as products of primes, i.e. $\exists p_1, \ldots p_\ell, q_1, \ldots q_m$, all prime, such that 
$$ a = p_1 \cdots p_\ell, \qquad b = q_1 \cdots q_m.$$

Then 
$$ k+1 = ab =  p_1 \cdots p_\ell \, q_1 \cdots q_m, $$
therefore we can write $k+1$ as a product of primes.

Thus the claim holds by strong induction.
\end{proof}

The Principle of Strong Mathematical Induction and the Principle of Mathematical Induction are logically equivalent, but sometimes it is easier to use one or the other, as we saw.


\subsection{Exercises}
\begin{enumerate}
\item Prove De Morgan's Laws for propositions: $\neg (P \wedge Q) = \neg P \vee \neg Q$ and $\neg (P \vee Q) = \neg P \wedge \neg Q$ (Hint: use truth tables).
\item Write the following statements and their negations using logical quantifier notation and then prove or disprove them:
\begin{enumerate}
    \item[(i)] Every odd integer is divisible by three.
    \item [(ii)] For any real number, twice its square plus twice itself plus six is greater than or equal to five. \textit{(You may assume knowledge of calculus.)}
    \item[(iii)] Every integer can be written as a unique difference  of two natural numbers.
\end{enumerate}
\item Prove the following statements:
\begin{enumerate}
    \item[(i)] If $a | b$ and $a,b \in \N$ (positive integers), then $a \leq b$.    \item[(ii)] If $a | b$ and $a | c$, then $a | (x b + y c)$, where $a,b,c,x,y \in \Z$.
    \item[(iii)] Let $a,b,n \in \Z$. If $n$ does not divide the product $ab$, then $n$ does not divide $a$ and $n$ does not divide $b$.
\end{enumerate}
\item Prove that for all integers $n \geq 1$, $3|(2^{2n}-1)$.
\item Prove the Fundamental Theorem of Arithmetic, that every integer $n \geq 2$ has a unique prime factorization (i.e. prove that the prime factorization from the last proof is unique).
\end{enumerate}


\subsection{References}
Most of this content may be found in Chapter 1 of \cite{proofs}, though many of the examples are my own. \cite{toolsreasoning} is also a great resource. In particular, the content on induction comes from there. Unfortunately the latter text is not freely available online or at U of T. 


\section{Set theory}

\subsection{Basics}

For our purposes, we define a \emph{set} to be a collection of mathematical objects. If $S$ is a set and $x$ is one of the objects in the set, we say $x$ is an element of $S$ and denote it by $x\in S$. The set of no elements is called empty set and is denoted by $\emptyset$.

\begin{definition}[Subsets, Union, Intersection]
Let $S, T$ be sets. 
\begin{itemize}
    \item We say that $S$ is a \emph{subset} of $T$, denoted $S\subseteq T$, if $s\in S$ implies $s\in T$. 
    \item We say that $S=T$ if $S\subseteq T$ and $T\subseteq S$.
    \item We define the \emph{union} of $S$ and $T$, denoted $S \cup T$, as all the elements that are in \emph{either} $S$ or $T$.
    \item We define the \emph{intersection} of $S$ and $T$, denoted $S \cap T$, as all the elements that are in \emph{both} $S$ and $T$.
    \item We say that $S$ and $T$ are \emph{disjoint} if $S \cap T = \emptyset$.
\end{itemize}
\end{definition}

\begin{example}
$\mathbb{N} \subseteq \mathbb{N}_0 \subseteq \mathbb{Z} \subseteq \mathbb{Q} \subseteq \mathbb{R} \subseteq \mathbb{C}$
\end{example}

\begin{example}  Let $a, b \in \R$ such that $a < b$. \\
Open interval: $(a,b) := \{x \in \R : a < x < b \}$  ($a,b$ may be $- \infty$ or $+ \infty$)\\
Closed interval: $[a,b] := \{x \in \R : a \leq x \leq b \}$ \\
We can also define half-open intervals.  
\end{example}

\begin{example}
Let $A = \{x \in \N: 3 | x \}$ and $B = \{x \in \N: 6 | x \}$
Show that $B \subseteq A$. 
\end{example}
\begin{proof}
Let $x \in B$. Then $6 |x$, i.e. $\exists j \in \Z$ such that $x = 6j$. Therefore $x = 3 (2j)$, so $3|x$. Thus $x \in A$.
\end{proof}

\begin{definition}
Let $A,B \subseteq X$. We define the \emph{set-theoretic difference} of $A$ and $B$, denoted $A \setminus B$ (sometimes $A-B$) as the elements of $X$ that are in $A$ but \emph{not} in $B$. 

The complement of a set $A \subseteq X$ is the set $A^c := X \setminus A$.
\end{definition}

We extend the definition of union and intersection to an arbitrary family of sets as follows:

\begin{definition}
Let $S_\alpha$, $\alpha \in A$, be a family of sets. $A$ is called the \emph{index set}. We define
\begin{equation*}
    \bigcup_{\alpha \in A} S_\alpha := \{ x: \exists \alpha \text{ such that } x \in S_\alpha \},
\end{equation*}
\begin{equation*}
    \bigcap_{\alpha \in A} S_\alpha := \{ x: x \in S_\alpha \text{ for all } \alpha \in A \}.
\end{equation*}
\end{definition}

\begin{example}
$$\bigcup_{n=1}^\infty [-n,n] = \R$$
$$\bigcap_{n=1}^\infty \left( -\frac{1}{n},\frac{1}{n} \right) = \{0 \}$$
\end{example}

\begin{theorem}[De Morgan's Laws]
Let $\{S_\alpha\}_{\alpha \in A}$ be an arbitrary collection of sets. Then 
\begin{equation*}
    \left( \bigcup_{\alpha \in A} S_\alpha \right)^c = \bigcap_{\alpha \in A}  S_\alpha^c \quad \text{and} \quad \left( \bigcap_{\alpha \in A} S_\alpha \right)^c = \bigcup_{\alpha \in A}  S_\alpha^c
\end{equation*}
\end{theorem}

\begin{proof}
For the first part: Let $x \in \left( \bigcup_{\alpha \in A} S_\alpha \right)^c$. This is true if and only if $x \notin \left( \bigcup_{\alpha \in A} S_\alpha \right)$, or in other words $x \in S_\alpha^c$ for all $\alpha \in A$. This is true if and only if $x \in \bigcap_{\alpha \in A} S_\alpha^c$, which gives the result. \\
The second part is similar and is left as an exercise. 
\end{proof}

Since a set is itself a mathematical object, a set can itself contain sets.
\begin{definition}
The power set $\cP(S)$ of a set $S$ is the set of all subsets of $S$.
\end{definition}

\begin{example}
Let $S = \{a,b,c\}$. Then $\cP(S) = \{ \emptyset, \{a\}, \{b\}, \{c\}, \{a,b\},\{b,c\}, \{a,c\}, S \}$. 
\end{example}

Another way of building a new set from two old ones is the Cartesian product of two sets.

\begin{definition}\label{def:cartes_prod}
Let $S,T$ be sets. The \emph{Cartesian product} $S\times T$ is defined as the set of tuples with elements from $S,T$, i.e 
\begin{equation*}
    S\times T = \{ (s,t) \; \colon \; s \in S \; \text{ and } \; t \in T\}.
\end{equation*}
\end{definition}

This can also be extended inductively to a finite family of sets. 

\subsection{Ordered sets}\label{sec:ordered_sets}

\begin{definition}
A \emph{relation} $R$ on a set $X$ is a subset of $X \times X$. We say that $x \leq y$ if $(x,y)\in \R$. A relation $\leq$ is called a \emph{partial order} on $X$ if it satisfies
\begin{enumerate}
\item Reflexivity: $x \leq x$ for all $x \in X$
\item Transitivity: for $x,y,z \in X$, $x \leq y$ and $y \leq z$ implies $x \leq z$
\item Anti-symmetry: for $x,y \in X$, $x \leq y$ and $y \leq x$ implies $x = y$
\end{enumerate}
The pair $(X, \leq)$ is called a \emph{partially ordered set}.

A \emph{chain} or \emph{totally ordered set} $C \subseteq X$ is a subset with the property $x \leq y$ or $y \leq x$ for any $x,y \in C$.
\end{definition}


\begin{example}
The real numbers with the usual ordering, $(\R, \leq)$, are totally ordered. 
\end{example}

\begin{example}
The power set of a set $X$ with the ordering given by subsets, $(\cP(X), \subseteq)$ is partially ordered set. 
\end{example}

\begin{example}
Let $X = \{a,b,c,d\}$. What is $\cP(X)$? Find a chain in $\cP(X)$.

$\cP(X) = \{\emptyset,\{a\},\{b\},\{c\},\{d\},\{a,b\},\{b,c\},\{c,d\},\{b,d\},\{a,c\},\{a,d\}, \{a,b,c\},\{b,c,d\},\{a,b,d\},\{a,c,d\},X\}$

An example of a chain $C \subseteq \cP(X)$ is $C = \{\emptyset,\{b\},\{b,c\}, \{a,b,c\},X\}$
\end{example}

\begin{example}
Consider the set $C([0,1],\R):= \{f:[0,1] \to \R : f \text{ is continuous}\}$.

For two functions $f,g \in C([0,1],\R)$, we define the ordering as $f \leq g$ if $f(x) \leq g(x)$ for $x \in [0,1]$. Then $(C([0,1],\R),\leq)$ is a partially ordered set. Can you think of a chain that is a subset of $(C([0,1],\R)$?
\end{example}

\begin{definition}
A non-empty partially ordered set $(X,\leq)$ is \emph{well-ordered} if every non-empty subset $A \subseteq X$ has a mimimum element.
\end{definition}

Recall that we already saw that $\N$ is well-ordered, as we used it to prove the principle of mathematical induction. $\R$ with the usual order does not have this property.

Having a partially ordered set allows us to talk about upper and lower bounds. 

\begin{definition}%Marcoux Definition 2.12
Let $(X,\leq)$ be a partially ordered set and $S\subseteq X$. Then $x\in X$ is an \emph{upper bound} for $S$ if for all $s \in S$ we have $x\leq s$. Similarly $y\in X$ is a \emph{lower bound} for $S$ if for all $s\in S$, $y\leq s$. If there exists an upper bound for $S$, we call $S$ \emph{bounded above} and if there exists a lower bound for $S$, we call $S$ \emph{bounded below}. If $S$ is bounded above and bounded below, we say $S$ is \emph{bounded}. 
\end{definition}

We can also ask if there exists a least upper bound or a greatest lower bound. 

\begin{definition}%Marcoux Definition 2.12
Let $(X,\leq)$ be a partially ordered set and $S\subseteq X$. We call $x\in X$ \emph{least upper bound} or \emph{supremum}, denoted $x= \sup S$, if $x$ is an upper bound and for any other upper bound $y\in X$ of $S$ we have $x\leq y$. Likewise $x\in X$ is the \emph{greatest lower bound} or \emph{infimum} for $S$, denoted $x= \inf S$, if it is a lower bound and for any other lower bound $y\in X$, $y\leq x$.
\end{definition}

Note that the supremum and infimum of a bounded set do not necessarily need to exist. However, if they do exists they are unique, which justifies the article \emph{the} (see Exercise 4). Nevertheless, the reals have a remarkable property, which we will take as an axiom.

\begin{axiom}\label{ax: completeness}[Completeness Axiom]
Let $S\subseteq \R$ be bounded above. Then there exists $r\in \R$ such that $r= \sup S$, i.e. $S$ has a least upper bound. 
\end{axiom}

By setting $S^\prime = -S:= \{ -s \; \colon \; s\in S\}$ and noting $\inf S = - \sup S^\prime$, we obtain a similar statement for infima if $S$ is bounded below. As mentioned above, this property is fairly special, for example it fails for the rationals.

\begin{example}
Let $S= \{q \in \Q \; \colon \; x^2 < 7 \}$. Then $S$ is bounded above in $\Q$, but there exists no least upper bound in $\Q$.
\end{example}

There is a nice alternative characterization for suprema in the real numbers. 

\begin{proposition}
 Let $S\subseteq \R$ be bounded above. Then $r= \sup S$ if and only if $r$ is an upper bound and for all $\epsilon>0$ there exists an $s\in S$ such that $r-\epsilon <s$. 
\end{proposition}

\begin{proof}
($\Rightarrow$) We will prove the forward direction ($\implies$) by contrapositive. Suppose $r$ is either not an upper bound or there exists an $\epsilon>0$ such that for all $s\in S$, $r-\epsilon \geq s$. In the first case, $r$ is not the supremum by definition. In the second case, $r-\varepsilon$ is an upper bound which is smaller than $r$. Thus $r\neq \sup S$. 

($\Leftarrow$) For the backward direction we will proceed by contradiction. Suppose $r$ is an upper bound and for all $\epsilon>0$ there exists an $s\in S$ such that $r-\epsilon <s$, but $r\neq \sup S$. Then $\sup S < r$ or equivalently $r-\sup S>0$. Then by assumption there exists an $s\in S$ such that $\sup S = r-(r-\sup S)<s$, which contradicts the definition of supremum.
\end{proof}

Using the same trick, we may obtain a similar result for infima.

\begin{proposition}
 Let $S\subseteq \R$ be bounded below. Then $r= \inf S$ if and only if $r$ is a lower bound and for all $\epsilon>0$ there exists an $s\in S$ such that $r+\epsilon >s$. 
\end{proposition}

\begin{example}
Consider $S = \{1/n \; \colon \; n\in \N\}$. Then $\sup S = 1$ and $\inf S = 0$.
\end{example}


\subsection{Functions}

One way to define a function is as follows :
\begin{definition}[{\cite[Definition 1.1.14]{tastetopology}}]
A function $f$ from a set $X$ to a set $Y$ is a subset of $X \times Y$ with the properties:
\begin{enumerate}
    \item For every $x \in X$, there exists a $y \in Y$ such that $(x,y) \in f$
    \item If $(x,y) \in f$ and $(x,z) \in f$, then $y = z$.
\end{enumerate}
$X$ is called the \emph{domain} of $f$.
\end{definition}
How does this connect to other descriptions of functions you may have seen? Instead of writing $f \subseteq X \times Y$, we often write $f: X \to Y, \; x \mapsto y$, where $(x,y) \in f$.

\begin{example}
For a set $X$, the identity function is:
$$ 1_X: X \to X, \quad x \mapsto x $$
\end{example}

\begin{definition}[Image and pre-image]
Let $f:X \to Y$ and $A \subseteq X$ and $B \subseteq Y$. The image of $f$ is the set $f(A) := \{f(x): x \in A \}$ and the pre-image of $f$ is the set $f^{-1}(B) := \{x: f(x) \in B \}$
\end{definition}

The following re-statements of the above may be helpful way to think about it for proofs: \\
If $y \in f(A)$, then $y \in Y$, and there exists an $x \in A$ such that $y = f(x)$. \\
If $x \in f^{-1}(B)$, then $x \in X$ and $f(x) \in B$.


\begin{definition}[Surjective, injective and bijective]
Let $f:X \to Y$, where $X$ and $Y$ are sets. Then
\begin{itemize}
    \item $f$ is \emph{injective} if $x_1 \neq x_2$ implies $f(x_1) \neq f(x_2)$
    \item $f$ is \emph{surjective} if for every $y \in Y$, there exists an $x \in X$ such that $y = f(x)$
    \item $f$ is \emph{bijective} if it is both injective and bijective
\end{itemize}
\end{definition}

\begin{example}
Let $f:X \to Y$, $ x \mapsto x^2$. \\
If $X = \R$ and $Y= [0,\infty)$: $f$ is surjective. \\
If $X = [0,\infty)$ and $Y = \R$: $f$ is injective. \\
If $X = Y = [0,\infty)$: $f$ is bijective. \\
If $X=Y=\R$, then $f$ is neither surjective nor injective.
\end{example}

\begin{proposition}
\label{prop:set_subset_preim_im}
Let $f: X \to Y$ and $A \subseteq X$. Prove that $A \subseteq f^{-1}(f(A))$, with equality if $f$ is injective. 
\end{proposition}
\begin{proof}
First we show $A \subseteq f^{-1}(f(A))$.
Let $x \in A$. Let $B = f(A)$, $B \subseteq Y$. By definition, $f(x) \in B$. So then again by definition, $x \in f^{-1}(B)$. Thus $x \in f^{-1}(f(A))$.

Next, suppose $f$ is injective. We have already shown that $A \subseteq f^{-1}(f(A))$, so it remains to show that  $f^{-1}(f(A)) \subseteq  A$. Let $x \in f^{-1}(f(A))$. Then $f(x) \in f(A)$ by the definition of the pre-image. This means that there exists a $\tilde x \in A$ such that $f(x) = f(\tilde x)$. Since $f$ is injective, we have $x = \tilde x$, and hence $x \in A$.
\end{proof}
% could make this an iff

\subsection{Cardinality}

Intuitively, the \emph{cardinality} of a set $A$, denoted $|A|$, is the number of elements in the set. For sets with only a finite number of elements, this intuition is correct. We call a set with finitely many elements finite. 

We say that the empty set has cardinality 0 and is finite.

\begin{proposition}
 If $X$ is finite set of cardinality $n$, then the cardinality of $\cP(X)$ is $2^n$.
\end{proposition}
\begin{proof}
We proceed by induction. First, suppose $n=0$. Then $X = \emptyset$, and $\cP(X) = \{ \emptyset \}$ which has cardinality $1 = 2^0$.

Next, suppose that the claim holds for some $n \in \N_0$. Let $X$ have $n+1$ elements. Let's call them $\{x_1, \ldots, x_n, x_{n+1}\}$. Then we can split $X$ up into subsets $A=\{x_1, \ldots, x_n\}$ and $B=\{ x_{n_1}\}$. By the inductive hypothesis, $\cP(A)$ has cardinality $2^n$. Any subset of $X$ must either be a subset of $A$ or contain $x_{n+1}$. How many subsets are there for the latter form? Let's count them out. Each subset will be formed by taking elements from $A$ and combining them with $x_{n+1}$. We start with no elements from $A$ and count up to all of them:
\begin{align*}
& 1 + \binom{n}{1} + \binom{n}{2} + \ldots + \binom{n}{n-1}  + \binom{n}{n} \\
=& \sum_{k=0}^n \binom{n}{k} \\
=& 2^n
\end{align*}
Therefore the total number of elements in $\cP(X)$ is the number of subsets of $A$ ($2^n$) plus the number of mixed subsets ($2^n$), i.e. the cardinality of $\cP(X)$ is $2^n+ 2^n = 2^{n+1}.$

Thus the claim holds by induction. 
\end{proof}

Note: you do not need to prove this by induction. It can also be proved combinatorially (exercise).

\begin{definition}
Two sets $A$ and $B$ have same cardinality, $|A| = |B|$, if there exists bijection $f:A \to B$.
\end{definition}

\begin{example}
Which is bigger, $\N$ or $\N_0$? \\
Intuitively, it seems that $\N_0$ should be bigger, since it includes exactly one more element than $\N$, namely 0. However, clearly the function $f:\N_0 \to \N$ defined by $n \mapsto n+1$ is a bijection. Therefore $\N_0$ and $\N$ have the same cardinality! One way to think about this is that $\N_0$ and $\N$ are the ``same size'' of infinity. 
\end{example}

It may sometimes be difficult to find such a bijection. However you can also use the following definition and theorem to instead show that two sets have the same cardinality by finding two injective functions between them.

\begin{definition}
We say that the cardinality of a set $A$ is less than the cardinality of a set $B$, denoted $|A| \leq |B|$ if there exists an injection $f:A \to B$.
\end{definition}

\begin{theorem}[Cantor-Schr\"{o}der-Bernstein]
Let $A$, $B$, be sets. If $|A| \leq |B|$ and $|B| \leq |A|$, then $|A| = |B|$.
\end{theorem}
Proof is omitted. See \cite[Theorem 1.2.7]{tastetopology}

\begin{example}
\label{ex:nbyn}
$|\N| = |\N \times \N|$
\end{example}
\begin{proof}
First, we show $|\N| \leq|\N \times \N|$. The function $f: \N \to \N \times \N$ defined by $n \mapsto (n,1)$ is an injection, thus $|\N| \leq|\N \times \N|$. 

Next, we show $|\N \times \N| \leq |\N|$. We define the function $g: \N \times \N \to \N$  by $(n,m) \mapsto 2^n 3^m$. Why is this an injection? Assume we have $n_1,n_2,m_1,m_2$ such that $2^{n_1}3^{m_1} = 2^{n_2}3^{m_2}$. We need to show $n_1 = n_2$ and $m_1 = m_2$.  By the Fundamental Theorem of Arithmetic, every natural number greater than 1 has a unique prime factorization, so therefore the result must hold.
\end{proof}


\begin{definition}
Let $A$ be a set. 
\begin{enumerate}
\item $A$ is \emph{finite} if there exists an $n \in \N$ and a bijection $f:\{1,\ldots,n\} \to A$
\item $A$ is \emph{countably infinite} if there exists a bijection $f:\N\to A$
\item $A$ is \emph{countable} if it is finite or countably infinite
\item $A$ is \emph{uncountable} otherwise
\end{enumerate}
\end{definition}

\begin{example}
The rational numbers are countable, and in fact $|\Q| = |\N|$. \\
Let's look at $\Q^+ := \{ x \in \Q : x > 0\}$. The fact that the rationals are countable relies on this famous way of listing the rational numbers:
\begin{equation*}
\scalebox{1.2}{$
 \begin{array}{cccccc}
 1 & \frac{1}{2} & \frac{1}{3} & \frac{1}{4} &\frac{1}{5} &\ldots \\[0.5em]
 2 & \textcolor{red}{\frac{2}{2}} & \frac{2}{3} &\textcolor{red}{\frac{2}{4}} &\frac{2}{5} & \ldots \\[0.5em]
 3 & \frac{3}{2} & \textcolor{red}{\frac{3}{3}} &\frac{3}{4} &\frac{3}{5} &\ldots \\[0.5em]
 4 & \textcolor{red}{\frac{4}{2}} & \frac{4}{3} & \textcolor{red}{\frac{4}{4}} &\frac{4}{5} &\ldots \\[0.5em]
 \vdots & \vdots & \vdots & \vdots &\vdots & \ddots 
\end{array}  $} 
\end{equation*}
This is a map from $\N$ to $\Q^+$. As long as we skip any fraction that is already in our list as we go along, it is injective.  Since we can find an injection from $\Q^+$ to $\N \times \N$ (take $q/p \mapsto (q,p)$), and $|\N| = |\N \times \N|$ by \cref{ex:nbyn}, we have that $|\Q^+| = |\N|$. 

We can extend this to $\Q$. To do so, let $f\colon \N \to \Q^+$ be a bijection (which exists by the previous part). Then we can define another bijection $g\colon \N \to \Q$ by setting $g(1) = 0$ and 
\begin{equation*}
    g(n) =\begin{cases}
    f(n) & \text{ if } n \text{ is even,} \\
    -f(n) & \text{ if }  n \text{ is odd},
    \end{cases}
\end{equation*}
for $n>1$.
\end{example}

Next we show that $\N$ is ``smaller'' than $(0,1)$.
\begin{theorem}
The cardinality of $\N$ is smaller than that of $(0,1)$.
\end{theorem}
\begin{proof}
First, we show that there is an injective map from $\N$ to $(0, 1)$. The map $n \to \frac{1}{n}$ fulfils this. 

Next, we show that there is no surjective map from $\N$ to (0, 1). We use the fact that every number $r \in (0,1)$ has a binary expansion of the form $r=0.\sigma_1\sigma_2\sigma_3\ldots$ where $\sigma_i \in \{0, 1\}$, $i \in \N$.

Now we suppose in order to derive a contradiction that there does exist a surjective map $f$ from $\N$ to (0, 1)., i.e. for $n \in \N$ we have $f(n) = 0.\sigma_1(n)\sigma_2(n)\sigma_3(n)\ldots$. This means we can list out the binary expansions, for example like
\begin{align*}
f(1)= & 0.\textcolor{red}{0}0000000\ldots \\
f(2)=& 0.1\textcolor{red}{1}11111111\ldots\\
f(3)=& 0.01\textcolor{red}{0}1010101\ldots  \\
f(4)= & 0.101\textcolor{red}{0}101010\ldots  \\
& 
\end{align*}

We will construct a number $\tilde r \in (0,1)$ that is not in the image of $f$. Define $\tilde r = 0.\tilde\sigma_1 \tilde\sigma_2 \ldots$, where we define the $n$th entry of $\tilde r$ to be the the opposite of the  $n$th entry of the $n$th item in our list:
\begin{equation*}
    \tilde\sigma_n = \begin{cases} 1 & \text{if } \sigma_n(n) = 0, \\
    0 & \text{if }  \sigma_n(n) = 1.
    \end{cases}
\end{equation*}
Then $\tilde r$ differs from $f(n)$ at least in the $n$th digit of its binary expansion for all $n\in \N$. Hence, $\tilde r\not\in f(\N)$, which is a contradiction to $f$ being surjective. This technique is often referred to as Cantor's diagonal argument. 
\end{proof}

\begin{proposition}
(0,1) and $\R$ have the same cardinality. 
\end{proposition}
\begin{proof}
The map $f:\R \to (0,1)$ defined by $x \mapsto \frac{1}{\pi} \left( \arctan(x) + \frac{\pi}{2} \right)$ is a bijection.
\end{proof}

We have shown that there are different sizes of infinity, as the cardinality of $\N$ is infinite but still smaller than that of $\R$ or $(0,1)$. In fact, we have
$$ |\N| = |\N_0| = |\Z| = |\Q| < | \R|.$$
Because of this, there are special symbols for these two cardinalities: The cardinality of $\N$ is denoted $\aleph_0$, while the cardinality of $\R$ is denoted $\mathfrak{c}$. 

\subsection{Exercises}
\label{subsec:ex_set_theory}
\begin{enumerate}
	\item Let $A = \{x\in \R : x <100\}$, $B = \{x\in \Z : |x| \geq 20\}$, and $C = \{y \in \N : y \text{ is prime}\}$ ($A,B,C \subseteq \R$). Find $A \cap B$, $B^c \cap C$, $B \cup C$, and $(A \cup B )^c$.
    \item Is $\R \times \R$ with the ordering $(x_1,y_1) \preceq (x_2,y_2)$ if $x_1 \leq x_2$ a partially ordered set? 
     \item \cite[Exercise 1.3.1]{tastetopology} Let $S$ be a non-empty set. A relation $R$ on $S$ is called an equivalence relation if it is
    \begin{enumerate}
        \item[(i)] Reflexive: $(x,x) \in R$ for all $x \in S$
        \item[(ii)] Symmetric: if $(x,y) \in R$  then $(y,x) \in R$ for all $x,y \in S$
        \item[(iii)] Transitive: if $(x,y), (y,z) \in R$ then $(x,z) \in R$ for all $x,y,z \in S$
    \end{enumerate}
Given $x \in S$, the equivalence class of $x$ (with respect to a given equivalence relation $R$) is defined to consist of those $y \in S$ for which $(x,y) \in R$. Show that two equivalence classes are either disjoint or identical.
    \item Let $(X, \leq)$ be a partially ordered set and $S\subseteq X$ be bounded. Show that the infimum and supremum of $S$ are unique (if they exist).
    \item Let $S,T \subseteq \R$ and suppose both are bounded above. Define $S+T = \{s + t \; \colon \; s\in S,t\in T\}$. Show that $S+T$ is bounded above and $\sup(S+T) = \sup S + \sup T$. 
    \item Let $f: X \to Y$, $X, y \subseteq \R$ be defined by the map $x \mapsto \sin(x)$. For what choices of $X$ and $Y$ is $f$ injective, surjective, bijective, or neither?
    \item Show that for sets $A,B \subseteq X$ and $f: X \to Y$, $f(A \cap B) \subseteq f(A) \cap f(B)$.
    \item Let $f: X \to Y$ and $B \subseteq Y$. Prove that $f(f^{-1}(B)) \subseteq B$, with equality iff $f$ is surjective.
    \item Prove that :
    \begin{enumerate}
        \item $f(\cup_{i \in I}A_i) = \cup_{i \in I}f(A_i)$, where $f:X \to Y$, $A_i \subseteq X \, \forall i \in I$
        \item $f^\inv(\cup_{i \in I}B_i) = \cup_{i \in I}f^\inv(B_i)$, where $f:X \to Y$, $B_i \subseteq Y \, \forall i \in I$
    \end{enumerate}
    \item Show that $\N$ and $\Z$ have the same cardinality.
    \item Show that $|(0,1)| =|(1,\infty)|$.
\end{enumerate}

%https://math.stackexchange.com/questions/359693/overview-of-basic-results-about-images-and-preimages

\subsection{References}
The content in this section mostly follows \cite{tastetopology}, but is supplemented by \cite{marcoux2019} (ordered sets) and \cite{piotr} (introductory set theory).


\newpage

\printbibliography

%\bibliographystyle{alphaurl} 
%\bibliography{references}

\end{document}
