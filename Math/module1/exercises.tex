\documentclass{article}
\usepackage[utf8]{inputenc}
\usepackage[margin=1in]{geometry}
\usepackage{hyperref}
\usepackage{setspace}
\pagenumbering{arabic}
\usepackage{graphicx}
\usepackage[dvipsnames]{xcolor}
\usepackage{fancyhdr} 
\usepackage{amsmath, amsfonts, amsthm, amssymb}
\usepackage{bbm}
\usepackage{nth}
\usepackage{dsfont}

\hypersetup{
  colorlinks   = true, %Colours links instead of boxes
  urlcolor     = black, %Colour for external hyperlinks
  linkcolor    = black, %Colour of internal links
  citecolor   = black %Colour of citations
}

\allowdisplaybreaks % fixes align environment weird spacing on page
\setlength{\parindent}{0cm}


%For references using Lemma, Theorem, etc, use \cref
\usepackage[nameinlink,capitalize,sort]{cleveref}

% theorems
\newtheorem{theorem}{Theorem}[section]
\newtheorem{lemma}[theorem]{Lemma}
\newtheorem{definition}[theorem]{Definition}
\newtheorem{proposition}[theorem]{Proposition}
\newtheorem{example}[theorem]{Example}
\newtheorem{corollary}[theorem]{Corollary}
%theoremstyle{plain} %boldface title, italicized body. Commonly used in theorems, lemmas, corollaries, propositions and conjectures.
%\theoremstyle{definition} %boldface title, Roman body. Commonly used in definitions, conditions, problems and examples
\theoremstyle{remark} %italicized title, Roman body. Commonly used in remarks, notes, annotations, claims, cases, acknowledgments and conclusions.
\newtheorem{exercise}[theorem]{Exercise}

\newenvironment{solution}
  {\renewcommand\qedsymbol{$\blacksquare$}\begin{proof}[Solution]}
  {\end{proof}}

% weird hack to get rid of dot:
\usepackage{xpatch}
\makeatletter
\AtBeginDocument{\xpatchcmd{\@thm}{\thm@headpunct{.}}{\thm@headpunct{}}{}{}}



% lin alg
\newcommand{\bu}{{\mathbf{u}}}
\newcommand{\bv}{{\mathbf{v}}}
\newcommand{\bw}{{\mathbf{w}}}
\newcommand{\bx}{\mathbf{x}}
\newcommand{\zerovec}{{\mathbf{0}}}

% other useful stuff
\newcommand{\Id}{{\mathds{1}}}
\newcommand{\R}{{\mathbb{R}}}
\newcommand{\C}{{\mathbb{C}}}
\newcommand{\Z}{{\mathbb{Z}}}
\newcommand{\N}{{\mathbb{N}}}
\newcommand{\F}{{\mathbb{F}}}
\newcommand{\cL}{{\mathcal{L}}}
\newcommand{\cP}{\mathcal{P}}
\newcommand{\cT}{\mathcal{T}}
\newcommand{\inv}{{-1}}

%\DeclareMathOperator{\dim}{dim}
\DeclareMathOperator{\range}{range}
\DeclareMathOperator{\rank}{rank}
\DeclareMathOperator{\nullity}{nullity}


\begin{document}
\begin{center}
\Large{Exercises for Module 1: Proofs}
\end{center}


1.  Prove De Morgan's Laws for propositions: $\neg (P \wedge Q) = \neg P \vee \neg Q$ and $\neg (P \vee Q) = \neg P \wedge \neg Q$ (Hint: use truth tables).

\vspace{11cm} % delete this

% \begin{proof}  
% ADD your content 
% \end{proof}  



2. If $a | b$ and $a,n \in \Z_{>0}$ (positive integers), then $a \leq b$.

\vspace{11cm} % delete this

% \begin{proof}  
% ADD your content 
% \end{proof}  


3. If $a | b$ and $a | c$, then $a | (x b + y c)$, where $x,y \in \Z$.

\vspace{12cm} % delete this

% \begin{proof}  
% ADD your content 
% \end{proof}  



4. Let $a,b,n \in \Z$. If $n$ does not divide the product $ab$, then $n$ does not divide $a$ and $n$ does not divide $b$.

\vspace{11cm} % delete this

% \begin{proof}  
% ADD your content 
% \end{proof}  



5. Prove that for all integers $n \geq 1$, $3|(2^{2n}-1)$.

\vspace{12cm} % delete this

% \begin{proof}  
% ADD your content 
% \end{proof}  



6. Prove the Fundamental Theorem of Arithmetic, that every integer $n \geq 2$ has a unique prime factorization (i.e. prove that the prime factorization from the last proof is unique).

\vspace{11cm} % delete this

% \begin{proof}  
% ADD your content 
% \end{proof}  



7. Let $A = \{x\in \R : x <100\}$, $B = \{x\in \Z : |x| \geq 20\}$, and $C = \{y \in \N : y \text{ is prime}\}$. Find $A \cap B$, $B^c \cap C$, $B \cup C$, and $(A \cup B )^c$.


\vspace{11cm} % delete this

% \begin{proof}  
% ADD your content 
% \end{proof}  




\end{document}


