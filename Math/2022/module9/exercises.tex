\documentclass{article}
\usepackage[utf8]{inputenc}
\usepackage[margin=1in]{geometry}
\usepackage{hyperref}
\usepackage{setspace}
\pagenumbering{arabic}
\usepackage{graphicx}
\usepackage[dvipsnames]{xcolor}
\usepackage{fancyhdr} 
\usepackage{amsmath, amsfonts, amsthm, amssymb}
\usepackage{bbm}
\usepackage{nth}
\usepackage{dsfont}

\hypersetup{
  colorlinks   = true, %Colours links instead of boxes
  urlcolor     = black, %Colour for external hyperlinks
  linkcolor    = black, %Colour of internal links
  citecolor   = black %Colour of citations
}

\allowdisplaybreaks % fixes align environment weird spacing on page
\setlength{\parindent}{0cm}


%For references using Lemma, Theorem, etc, use \cref
\usepackage[nameinlink,capitalize,sort]{cleveref}

% theorems
\newtheorem{theorem}{Theorem}[section]
\newtheorem{lemma}[theorem]{Lemma}
\newtheorem{definition}[theorem]{Definition}
\newtheorem{proposition}[theorem]{Proposition}
\newtheorem{example}[theorem]{Example}
\newtheorem{corollary}[theorem]{Corollary}
%theoremstyle{plain} %boldface title, italicized body. Commonly used in theorems, lemmas, corollaries, propositions and conjectures.
%\theoremstyle{definition} %boldface title, Roman body. Commonly used in definitions, conditions, problems and examples
\theoremstyle{remark} %italicized title, Roman body. Commonly used in remarks, notes, annotations, claims, cases, acknowledgments and conclusions.
\newtheorem{exercise}[theorem]{Exercise}

\newenvironment{solution}
  {\renewcommand\qedsymbol{$\blacksquare$}\begin{proof}[Solution]}
  {\end{proof}}

% weird hack to get rid of dot:
\usepackage{xpatch}
\makeatletter
\AtBeginDocument{\xpatchcmd{\@thm}{\thm@headpunct{.}}{\thm@headpunct{}}{}{}}



% lin alg
\newcommand{\bu}{{\mathbf{u}}}
\newcommand{\bv}{{\mathbf{v}}}
\newcommand{\bw}{{\mathbf{w}}}
\newcommand{\bx}{{\mathbf{x}}}
\newcommand{\by}{{\mathbf{y}}}
\newcommand{\bz}{{\mathbf{z}}}
\newcommand{\zerovec}{{\mathbf{0}}}
\newcommand{\innerprod}[1]{\langle #1 \rangle}
\newcommand{\Tr}{\mathrm{Tr}}

\DeclareMathOperator{\range}{range}
\DeclareMathOperator{\rank}{rank}
\DeclareMathOperator{\nullspace}{null}
\DeclareMathOperator{\nullity}{nullity}

% other useful stuff
\newcommand{\Id}{{\mathds{1}}}
\newcommand{\R}{{\mathbb{R}}}
\newcommand{\C}{{\mathbb{C}}}
\newcommand{\Z}{{\mathbb{Z}}}
\newcommand{\N}{{\mathbb{N}}}
\newcommand{\Q}{{\mathbb{Q}}}
\newcommand{\F}{{\mathbb{F}}}
\newcommand{\cL}{{\mathcal{L}}}
\newcommand{\cP}{\mathcal{P}}
\newcommand{\cT}{\mathcal{T}}
\newcommand{\inv}{{-1}}

% set theory
\newcommand{\interior}{\accentset{\circ}}



\begin{document}
\begin{center}
\Large{Exercises for Module 9: Linear Algebra III}
\end{center}


1.  Let $A,B \in M_n(\F)$ be similar matrices. Show that their characteristic polynomials coincide.

\vspace{11cm} % delete this

% \begin{proof}  
% ADD your content 
% \end{proof}  



2. Show that $A\in M_n(\C)$ is invertible if and only if $0 \not \in \sigma(A)$.

\vspace{11cm} % delete this

% \begin{proof}  
% ADD your content 
% \end{proof}  


3. Suppose $N$ is a nilpotent matrix. Show that $\sigma(N) = \{0\}$.

\vspace{12cm} % delete this

% \begin{proof}  
% ADD your content 
% \end{proof}  



4. Let $A\in M_n(\C)$ be an invertible matrix. Show that $\lambda$ is an eigenvalue of $A$ if and only if $\lambda^\inv$ is an eigenvalue of $A^\inv$.
\vspace{11cm} % delete this

% \begin{proof}  
% ADD your content 
% \end{proof}  



5. Suppose $A\in M_n(\C)$ is Hermitian. Show that all the eigenvalues of $A$ are real. Hint: Note that if $\bx$ is a normalized eigenvector of $A$ with eigenvalue $\lambda$, then $\innerprod{A\bx,\bx} = \lambda$.

\vspace{12cm} % delete this

% \begin{proof}  
% ADD your content 
% \end{proof}  



6. Let $A\in M_n(\R)$. Show that the eigenvalues of $A^TA$ are non-negative.
\vspace{11cm} % delete this

% \begin{proof}  
% ADD your content 
% \end{proof}  




\end{document}


