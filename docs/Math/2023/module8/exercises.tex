\documentclass{article}
\usepackage[utf8]{inputenc}
\usepackage[margin=1in]{geometry}
\usepackage{hyperref}
\usepackage{setspace}
\pagenumbering{arabic}
\usepackage{graphicx}
\usepackage[dvipsnames]{xcolor}
\usepackage{fancyhdr} 
\usepackage{amsmath, amsfonts, amsthm, amssymb}
\usepackage{bbm}
\usepackage{nth}
\usepackage{dsfont}

\hypersetup{
  colorlinks   = true, %Colours links instead of boxes
  urlcolor     = black, %Colour for external hyperlinks
  linkcolor    = black, %Colour of internal links
  citecolor   = black %Colour of citations
}

\allowdisplaybreaks % fixes align environment weird spacing on page
\setlength{\parindent}{0cm}


%For references using Lemma, Theorem, etc, use \cref
\usepackage[nameinlink,capitalize,sort]{cleveref}

% theorems
\newtheorem{theorem}{Theorem}[section]
\newtheorem{lemma}[theorem]{Lemma}
\newtheorem{definition}[theorem]{Definition}
\newtheorem{proposition}[theorem]{Proposition}
\newtheorem{example}[theorem]{Example}
\newtheorem{corollary}[theorem]{Corollary}
%theoremstyle{plain} %boldface title, italicized body. Commonly used in theorems, lemmas, corollaries, propositions and conjectures.
%\theoremstyle{definition} %boldface title, Roman body. Commonly used in definitions, conditions, problems and examples
\theoremstyle{remark} %italicized title, Roman body. Commonly used in remarks, notes, annotations, claims, cases, acknowledgments and conclusions.
\newtheorem{exercise}[theorem]{Exercise}

\newenvironment{solution}
  {\renewcommand\qedsymbol{$\blacksquare$}\begin{proof}[Solution]}
  {\end{proof}}

% weird hack to get rid of dot:
\usepackage{xpatch}
\makeatletter
\AtBeginDocument{\xpatchcmd{\@thm}{\thm@headpunct{.}}{\thm@headpunct{}}{}{}}



% lin alg
\newcommand{\bu}{{\mathbf{u}}}
\newcommand{\bv}{{\mathbf{v}}}
\newcommand{\bw}{{\mathbf{w}}}
\newcommand{\bx}{{\mathbf{x}}}
\newcommand{\by}{{\mathbf{y}}}
\newcommand{\bz}{{\mathbf{z}}}
\newcommand{\zerovec}{{\mathbf{0}}}
\newcommand{\innerprod}[1]{\langle #1 \rangle}
\newcommand{\Tr}{\mathrm{Tr}}


% other useful stuff
\newcommand{\Id}{{\mathds{1}}}
\newcommand{\R}{{\mathbb{R}}}
\newcommand{\C}{{\mathbb{C}}}
\newcommand{\Z}{{\mathbb{Z}}}
\newcommand{\N}{{\mathbb{N}}}
\newcommand{\Q}{{\mathbb{Q}}}
\newcommand{\F}{{\mathbb{F}}}
\newcommand{\cL}{{\mathcal{L}}}
\newcommand{\cP}{\mathcal{P}}
\newcommand{\cT}{\mathcal{T}}
\newcommand{\inv}{{-1}}

%\DeclareMathOperator{\dim}{dim}
\DeclareMathOperator{\range}{range}
\DeclareMathOperator{\rank}{rank}
\DeclareMathOperator{\nullspace}{null}
\DeclareMathOperator{\nullity}{nullity}


\begin{document}
\begin{center}
\Large{Exercises for Module 8: Linear Algebra II}
\end{center}


1. Let $D \in \mathcal{L}(\mathbb{P}_4(\R),\mathbb{P}_3(\R))$ be the differentiation map, $Dp = p'$. Find bases of $\mathbb{P}_4(\R)$ and $\mathbb{P}_3(\R)$ such that the matrix representation of $\mathcal{M}(D)$ with respect to these basis is given by 
    \begin{equation*}
    \mathcal{M}(D) = \begin{pmatrix}
    1 & 0 & 0 & 0 & 0 \\
    0 & 1 & 0 & 0 & 0 \\
    0 & 0 & 1 & 0 & 0 \\
    0 & 0 & 0 & 1 & 0
     \end{pmatrix}.
    \end{equation*}

\vspace{11cm} % delete this

% \begin{proof}  
% ADD your content 
% \end{proof}  



2. Show that matrix multiplication of square matrices is not commutative, i.e find matrices $A,B \in M_{2}$ such that $AB\neq BA$.
\vspace{11cm} % delete this

% \begin{proof}  
% ADD your content 
% \end{proof}  



3.  A square matrix is called \emph{nilpotent} if $\exists k \in \N$ such that $A^k = 0$. Show that for a nilpotent matrix $A$, $|A| = 0$.

\vspace{7cm} % delete this

% \begin{proof}  
% ADD your content 
% \end{proof}  



4. A real square matrix $Q$ is called \emph{orthogonal} if $Q^T Q = I$. Prove that if $Q$ is orthogonal, then $|Q| = \pm 1$.

\vspace{7cm} % delete this

% \begin{proof}  
% ADD your content 
% \end{proof}  

5. An $n \times n$ matrix is called \emph{antisymmetric} if $A^T = -A$. Prove that if $A$ is antisymmetric and $n$ is odd, then $|A|=0$. 

\vspace{7cm} % delete this

% \begin{proof}  
% ADD your content 
% \end{proof}  



6. Let $V$ be an inner product space, $U$ a vector space and $S\colon U \to V$, $T\colon U \to V$ be linear maps. Show that $\innerprod{S\bu,\bv}= \innerprod{T\bu, \bv}$ for all $\bu \in U$ and $\bv \in V$ implies $S=T$.

\vspace{10cm} % delete this

% \begin{proof}  
% ADD your content 
% \end{proof}  


7. Let $U,V,W$ be inner product spaces and $S,T \in \mathcal{L}(U,V)$ and $R\in \mathcal{L}(V,W)$. Show that the following holds
\begin{enumerate}
    \item $(S+\alpha T)^* = S^* + \overline{\alpha}T^*$ for all $\alpha\in \F$
    \item $(S^*)^* = S$
    \item $(RS)^* = S^*R^*$
    \item $I^* = I$, where $I \colon U \to U$ is the identity operator on $U$
\end{enumerate}

\vspace{11cm} % delete this




\newpage
\textcolor{white}{extra page}


\end{document}


