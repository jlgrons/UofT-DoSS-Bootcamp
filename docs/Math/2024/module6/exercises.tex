\documentclass{article}
\usepackage[utf8]{inputenc}
\usepackage[margin=1in]{geometry}
\usepackage{hyperref}
\usepackage{setspace}
\pagenumbering{arabic}
\usepackage{graphicx}
\usepackage[dvipsnames]{xcolor}
\usepackage{fancyhdr} 
\usepackage{amsmath, amsfonts, amsthm, amssymb}
\usepackage{bbm}
\usepackage{nth}
\usepackage{dsfont}

\hypersetup{
  colorlinks   = true, %Colours links instead of boxes
  urlcolor     = black, %Colour for external hyperlinks
  linkcolor    = black, %Colour of internal links
  citecolor   = black %Colour of citations
}

\allowdisplaybreaks % fixes align environment weird spacing on page
\setlength{\parindent}{0cm}


%For references using Lemma, Theorem, etc, use \cref
\usepackage[nameinlink,capitalize,sort]{cleveref}

% theorems
\newtheorem{theorem}{Theorem}[section]
\newtheorem{lemma}[theorem]{Lemma}
\newtheorem{definition}[theorem]{Definition}
\newtheorem{proposition}[theorem]{Proposition}
\newtheorem{example}[theorem]{Example}
\newtheorem{corollary}[theorem]{Corollary}
%theoremstyle{plain} %boldface title, italicized body. Commonly used in theorems, lemmas, corollaries, propositions and conjectures.
%\theoremstyle{definition} %boldface title, Roman body. Commonly used in definitions, conditions, problems and examples
\theoremstyle{remark} %italicized title, Roman body. Commonly used in remarks, notes, annotations, claims, cases, acknowledgments and conclusions.
\newtheorem{exercise}[theorem]{Exercise}

\newenvironment{solution}
  {\renewcommand\qedsymbol{$\blacksquare$}\begin{proof}[Solution]}
  {\end{proof}}

% weird hack to get rid of dot:
\usepackage{xpatch}
\makeatletter
\AtBeginDocument{\xpatchcmd{\@thm}{\thm@headpunct{.}}{\thm@headpunct{}}{}{}}



% lin alg
\newcommand{\bu}{{\mathbf{u}}}
\newcommand{\bv}{{\mathbf{v}}}
\newcommand{\bw}{{\mathbf{w}}}
\newcommand{\bx}{\mathbf{x}}
\newcommand{\zerovec}{{\mathbf{0}}}

% other useful stuff
\newcommand{\Id}{{\mathds{1}}}
\newcommand{\R}{{\mathbb{R}}}
\newcommand{\C}{{\mathbb{C}}}
\newcommand{\Z}{{\mathbb{Z}}}
\newcommand{\N}{{\mathbb{N}}}
\newcommand{\Q}{{\mathbb{Q}}}
\newcommand{\F}{{\mathbb{F}}}
\newcommand{\cL}{{\mathcal{L}}}
\newcommand{\cP}{\mathcal{P}}
\newcommand{\cT}{\mathcal{T}}
\newcommand{\inv}{{-1}}

%\DeclareMathOperator{\dim}{dim}
\DeclareMathOperator{\range}{range}
\DeclareMathOperator{\rank}{rank}
\DeclareMathOperator{\nullity}{nullity}


\begin{document}
\begin{center}
\Large{Exercises for Module 6: Metric Spaces IV}
\end{center}

1. Let $X$ be a set and define $d\colon X \times X \to [0,\infty)$ by 
    \begin{align*}
        d(x,y) = \begin{cases}
            0, & \; x=y, \\
            1, & \; x\neq y.
        \end{cases}
    \end{align*}
    Show that $S\subseteq X$ is compact if and only if $S$ is finite.
    
    \vspace{10.5cm} % delete this


2. Let $(X, d)$ be a metric space and $K \subset X$ compact. Show that for all $\epsilon>0$ there exists $\{x_1,x_2,\ldots, x_n\}\subseteq K$  such that for all $y\in K$ we have $d(y,x_i)<\epsilon$ for some $i=1,\ldots,n$.

\vspace{11cm} % delete this


3. Define the sequence $(a_n)_{n\in \N}$ by $a_1 =2$ and 
    \begin{equation*}
        a_{k+1} = \frac{a_k+5}{3}, \qquad k\geq 1.
    \end{equation*}
    Determine if the limit $\lim_{n\to \infty} a_n$ exists and, if so, calculate it.

\vspace{21cm} % delete this


4.   Let $(x_n)_{n\in \N}, (y_n)_{n\in \N}$ be bounded sequences in $\R$. Show that 
    \begin{align*}
        \limsup_{n\to \infty} (x_n + y_n) \leq \limsup_{n\to \infty} x_n + \limsup_{n\to \infty} y_n 
    \end{align*}
    and give an example where the inequality is strict. 

\vspace{10.5cm} % delete this

% \begin{proof}  
% ADD your content 
% \end{proof}  
5. Let $(x_n)_{n\in N}$ be a sequence in $\R$. Show that $\lim_{n\to \infty} x_n = 0$ if and only if $\limsup_{n\to \infty} \vert x_n\vert = 0$.

\vspace{11cm} % delete this


% \begin{proof}  
% ADD your content 
% \end{proof}  




\end{document}


